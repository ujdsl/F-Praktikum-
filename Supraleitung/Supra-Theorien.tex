\section{Theoretische Grundlagen}
\subsection{London-Theorie}

Die London-Theorie besteht im Kern aus zwei charakteristischen Gleichungen, welche das ohmsche Gesetz für Supraleiter ersetzen. Diese beiden Gleichungen wurden von den Brüdern Fritz und Heinz London 1935 aufgestellt. Experimentell lässt sich der Verlauf des Magnetfelds innerhalb eines Supraleiters über den Meißner-Ochsenfeld-Effekt beschreiben. Aufgrund dieses sollte das Innere nämlich feldfrei sein, was sich aber experimentell nicht bestätigen lässt.

Die Brüder London postulierten folgendes Gesetz:

\begin{align}
\overrightarrow{j}=\frac{nq\hslash}{m}\overrightarrow{\nabla}S - \frac{nq^{2}}{m}\overrightarrow{A}
\end{align}

Dabei bezeichnen S die Phase der makroskopischen Wellenfunktion und A das Vektorpotential. Die beiden London-Gleichungen erhält man nun durch Umformen der Gleichung zu:

\begin{align}
\partial_t \overrightarrow{j}=\frac{nq^2}{m}\overrightarrow{E}
\end{align}

\begin{align}
rot \overrightarrow{j}=-\frac{nq^2}{m}\overrightarrow{B}
\end{align}

Mit den London-Gleichungen und den Maxwell-Gleichungen lässt sich ein exponentiell Abklingendes Magnetfeld im Inneren des Supraleiters berechnen, welches experimentell bestätigt werden kann. 



\subsection{Ginzburg-Landau-Theorie}

Die Ginzburg-Landau-Theorie stellt eine makroskopische Theorie der Supraleitung über die Theorie der Phasenübergänge zweiter Ordnung dar. Gemäß dieser lässt sich die Freie Energie eines Supraleiters über einen komplexem Ordnungsparameter $\\psi$ ausdrücken:

\begin{align}
F=F_n + \alpha \vert\psi\vert^2 +\frac{\beta}{2}\vert\psi\vert^4+\frac{1}{2m*}\vert\left(\frac{\hslash}{i}\nabla +qA\right)\psi\vert^2 + \frac{\vert B \vert^2}{2\mu_0}
\end{align}

Die beiden Parameter $\alpha$ und $\beta$ sind dabei phämomenologische Parameter, welche für jedes Material neu bestimmt werden müssen. Aus der Minimierung der freien Energie kann man wiederum Gleichungen für den Ordnungsparameter und bspw. die Stromdichte herleiten, welche in enger Beziehung zu den London-Gleichungen stehen. 

Aus der Ginzburg-Landau-Theorie lassen sich zwei charakteristische Größen eines jeden Supraleiters herleiten. Diese sind erstens die Kohärenzlänge $\xi$, welche eine Größe für die Reichweite von Fluktuationen innerhalb der supraleitenden Phase ist.

\begin{align}
\xi= \sqrt{\frac{\hslash^2}{2m*\vert\alpha\vert}}
\end{align}

Weiterhin lässt sich auch die Eindringtiefe $\lambda$ des Magnetfelds in einen Supraleiter über die Ginzburg-Landau-Parameter angeben:

\begin{align}
\lambda=\sqrt{\frac{m*}{4\mu_0e^2\psi_0^2}}
\end{align}


\subsection{BCS-Theorie}

Die BCS-Theorie wurde 1957 von Bardeen, Cooper und Shrieffer gefunden und stellt die mikroskopische Erklärung der Supraleitung dar. Ausschlaggebend ist eine durch die Gitterschwingungen vermittelte, attraktive Wechselwirkung zwischen den Elektronen eines Kristallgitters. Diese kommt bei extrem niedrigen Temperaturen zum Tragen, da thermische Anregungen quasi nicht mehr existieren. Durch diese Wechselwirkung bilden sich aus zwei Elektronen sogenannte Cooper-Paare, welche die zweifache Elektronenmasse und -ladung aufweisen. Da hier zwei Elektronen mit Spin $\frac{1}{2}$ koppeln, besitzt das resultierende Cooper-Paar stringenterweise einen geradzahligen Spin und stellt damit ein Boson dar, für welches natürlich nicht mehr die Fermi-Dirac-Statistik, sondern die Bose-Einstein-Statistik gilt. Dies ist besonders ausschlaggebend für die Supraleitung, denn nun können alle gebildeten Cooper-Paare im gleichen Energiezustand vorliegen. Durch diesen makroskopisch besetzten Quantenzustand ergibt sich die widerstandslose Bewegung der Cooper-Paare durch das Kristallgitter und damit die Supraleitung. 

Will man nun solche gebildeten Cooper-Paare wieder aufbrechen, muss dem System Energie zugefügt werden. Experimentell beobachtet man jedoch kein kontinuierliches Aufbrechen der Cooper-Paare, sondern diese sind bis zu einer bestimmten zugeführten Energie stabil. Man spricht hier von einer sogenannten Energielücke, welches das Anregungsspektrum des Supraleiters aufweist. In dieser Energielücke existieren, wie der Name schon sagt, keine Zustände für die Elektronen. Die Energie des makroskopischen Quantenzustands und damit der Cooper-Paare liegt unterhalb der Energielücke, während die "normalen" Elektronenzustände alle überhalb der Energielücke liegen. Man muss einem supraleitenden System also immer eine Mindestenergie zuführen, um die Supraleitung und die Cooper-Paare aufbrechen zu können.
