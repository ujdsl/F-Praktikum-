\section{Zielsetzung}
Im Versuch "Supraleitung und Phasenübergänge" des F-Praktikums am PI soll das faszinierende Phänomen der Supraleitung qualitativ und quantitativ untersucht werden. Neben einem Demonstrationsversuch des Meißner-Ochsenfeld-Effekts an einem Hochtemperatursupraleiter wird der Phasenübergang einer Indium-Probe anhand des elektrischen Widerstands genau vermessen. Hierfür kommt als Kühlmittel unter anderem flüssiges Helium zum Einsatz, da die benötigten Temperaturen nicht mit flüssigem Stickstoff zu erreichen sind. Weiterhin wird das Phasendiagramm der Indium-Probe aufgenommen und die kritische Feldstärke gemessen. Als Zusatz wird die Hartshone-Spulenanordnung für die Messung der AC-Suszeptibilität verwendet. Wir erhoffen uns einige aussagekräftige Ergebnisse sowohl auf phänomenologische als auch quantitative Art. 