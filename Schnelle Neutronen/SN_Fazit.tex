\section{Zusammenfassung/Fazit}
Im durchgeführten Versuch konnten die enstehenden Neutronen eine Am-Be-Quelle sehr gut untersucht werden. Nach anfänglichen Schwierigkeiten mit der Elektronik konnte so doch noch durchaus zufriedenstellende Ergebnisse erreicht werden. Die Kernradienbestimmung und die Bestimmung der Radiuskonstante hat sehr gut funktioniert und sehr passable Werte geliefert, während die Energiebestimmung über die De-Broglie-Wellenlänge nicht ins physikalische Bild passt, wobei die Fehler noch teilweise unklar sind. Insgesamt konnten jedoch sehr gute Erkenntnisse über die Arbeit mit Neutronenquellen und ihre Anwendungen gewonnen werden.