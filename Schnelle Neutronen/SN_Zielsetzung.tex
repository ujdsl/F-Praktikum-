\section{Zielsetzung}

In diesem Versuch soll einerseits mit einer Time-of-FLight-Messung die maximale kinetische Energie von Neutronen, die aus einer $^{241}$Am-$^9$Be-Quelle emittiert werden, bestimmt werden. Außerdem werden die Neutronen dazu benutzt die Kernradien von verschiedenen Materialien zu bestimmen. Dies gelingt dadurch, dass die Neutronen mit den Elektronen in der Atomhülle aufgrund ihrer elektrischen Neutralität nicht wechselwirken. Der Effekt auf die Elektronen durch das geringe magnetische Moment des Neutrons kann vernachlässigt werden. 

Detektiert werden die Neutronen mit 2 organischen Szintillatoren, die sowohl für energieaufgelöste Messung, als auch für die Time-of-Flight-Messung geeignet sind. Für die Energie- und Zeitkalibrierung werden daher zwei weitere kleine Versuche durchgeführt. Für die Energiekalibrierung wird die Elektron-Positron-Annihilation genutzt, die nach dem $\beta$-Zerfall von $^{22}$Na stattfindet. Die Zeitkalibrierung wird mit Hilfe eines elektrischen Delays bei einer Vergleichsmessung durchgeführt.
