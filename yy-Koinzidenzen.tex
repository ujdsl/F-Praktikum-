\documentclass[11pt]{scrartcl}

\usepackage{ucs}
\usepackage[utf8x]{inputenc}
\usepackage[T1]{fontenc}
\usepackage[ngerman]{babel}
\usepackage{amsmath,amssymb,amstext}
\usepackage{graphicx}
\usepackage[justification=RaggedRight, singlelinecheck=false]{caption}


\title{Fortgeschrittenen Praktikum Teil 1: IKP}
\subtitle{Versuch 1: $\gamma \gamma$ - Koinzidenzen}

\author{Gruppe 1: Reinhold Kaiser, Florian Stoll}

\date{14.04.2018}

\begin{document}
\maketitle
\newpage
\tableofcontents
\newpage

\section{Einleitung}

\subsection{Zielsetzung}


\subsection{Theoretische Grundlagen}

Text 

\begin{table}
	\caption{Beispieltabelle}
	\begin{tabular}{|l|l|l|l|}
	\hline
	 & Netzteil & Multimeter & Oszilloskop \\ \hline
	Spannung & $5V$ & $ 5,076 V $ & $ 5,19 V $ \\ \hline
	Abweichung & $0\%$ & $1,5\%$ & $3,8\%$ \\ \hline
	\end{tabular}
\label{tab1}
\end{table}

\section{Versuchsaufbau und Messgeräte}

\begin{align}
U_{eff} = \frac{U_0}{\sqrt{2}} = \frac{\frac{U_{PeakPeak}}{2}}{\sqrt{2}} = \frac{17,4 V}{ 2 \sqrt{2}} \approx 6,15 V
\end{align}


%\begin{figure}[htbp] 
%     \includegraphics[width=0.5\textwidth]{Lissajous.jpg}
%  \caption{Beispielfigur}
%  \label{Lissajous}
%\end{figure}


\section{Versuchsteil 1}


\subsection{Durchführung}


\subsection{Auswertung}


\section{Versuchsteil 2}



\subsection{Durchführung}



\subsection{Auswertung}




\section{Zusammenfassung/Fazit}

Text


\end{document}