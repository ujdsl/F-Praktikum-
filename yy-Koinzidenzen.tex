\documentclass[11pt]{scrartcl}

\usepackage{ucs}
\usepackage[utf8x]{inputenc}
\usepackage[T1]{fontenc}
\usepackage[ngerman]{babel}
\usepackage{amsmath,amssymb,amstext}
\usepackage{graphicx}
\usepackage[justification=RaggedRight, singlelinecheck=false]{caption}


\title{Fortgeschrittenen Praktikum Teil 1: IKF}
\subtitle{Versuch 1: $\gamma \gamma$ - Koinzidenzen \\ Betreuer: Alexander Schottelius}

\author{Gruppe 1: Reinhold Kaiser, Florian Stoll}

\date{14.04.2018}

\begin{document}
\maketitle
\newpage
\tableofcontents
\newpage

\section{Zielsetzung}
In dem Versuch $\gamma$-$\gamma$-Koinzidenzen sollen für zwei Beispiele die Winkelverteilung der Zählraten
gemessen werden. Zum Einen entsteht beim Betazerfall von $^{22}Na$ ein Positron, welches wiederum mit
einem Elektron zu zwei $\gamma$-Quanten annihiliert. Zum Anderen zerfällt $^{60}Co$ zu $^{60}Ni$
und dieses dann über Kaskaden in den Grundzustand. Dabei werden zwei $\gamma$-Quanten emittiert,
deren Winkelkorrelation gemessen werden soll. Physikalisch relevant sind die Messungen von 
Winkelverteilungen, weil durch sie auf die Spin- und Drehimpulsquantenzahlen der angeregten Zustände 
geschlossen werden kann.


\section{Theoretische Grundlagen}

\subsection{Betazerfall von $^{22}Na$ und $^{60}Co$}

\subsection{Wechselwirkungen elektromagnetischer Strahlung mit Materie}

\paragraph{Der Photo-Effekt}



\paragraph{Der Compton-Effekt}



\subsection{Die Paarvernichtung}
Wie jedes Elementarteilchen besitzt auch das Elektron ein Anti-Teilchen mit gleichem Spin und gleicher Ruhemasse, jedoch mit entgegengesetzter Ladung. Dieses wird als Positron bezeichnet. Als Formelzeichen verwendet man hier $ e^{+} $ um die positive Ladung kennzuzeichnen. Dieses Positron kann als Nebenprodukt bei radioaktiven Atomkernen ( $\beta {^+} $ - Zerfall) oder durch die Paarbildung eines $\gamma $- Quants, das in ein Positron und eine Elektron zerfällt, entstehen. 
Die sogenannte Paarvernichtung ist der Umkehrprozess zur Paarbildung. Dabei wird ein Elektron-Positron-Paar wieder in elektromagnetische Strahlung umgewandelt. Dabei können jedoch unterschiedliche Anzahl an Photonen entstehen. Auch hier muss aber ein Stoß vorliegen, da sonst Impuls- und Energieerhaltung verletzt werden würden. Ein sogenannter Ein-Quanten-Zerfall, bei dem ein Paar tatsächlich nur in ein Photon umgewandelt wird, kann aus den gleichen Gründen nur bei einem Stoß mit einem Impuls- und Energieaufnahmefähigen Partner vorkommen, was nur im Festkörper vorkommt. Dieser Zerfall ist allerdings sehr unwahrscheinlich, und wird für diesen Versuch auch nicht berücksichtigt. 
Da das Elektron und das Positron beide den Spin $\frac{1}{2}$ haben, gibt es zwei verschiedene Einstellmöglichkeiten der Spins zueinander. Liegen die Spins parallel, besitzt das Paar einen Gesamtspin von 1, liegen sie anti-parallel ergibt sich ein Gesamtspin von 0. Der Grundzustand mit $S=0$ ist dabei nicht entartet, der Zustand mit $S=1$ ist dreifach entartet. Daraus ergibt sich eine Wahrscheinlichkeit von 3:1 für einen Zerfall des Triplett-Zustands, da die Wahrscheinlichkeiten für die Einstellmgölichkeiten der beiden Spins alle gleich sind. Die bei der Paarvernichtung entstehenden Quanten haben allerdings den Spin 1. Darum kann der Grundzustand nur in zwei $\gamma$ - Quanten zerfallen, während der Triplett-Zustand in drei Photonen zerfallen muss. Das stellt allerdings einen Prozess höherer Ordnung dar und ist daher deutlich unwahrscheinlicher. Im durchgeführten Versuch werden wir uns auch ausschließlich auf Paarvernichtungen aus dem Singulett - Zustand, d.h. mit zwei auftretenden $\gamma$ - Quanten beschränken.

\begin{table}
	\caption{Beispieltabelle}
	\begin{tabular}{|l|l|l|l|}
	\hline
	 & Netzteil & Multimeter & Oszilloskop \\ \hline
	Spannung & $5V$ & $ 5,076 V $ & $ 5,19 V $ \\ \hline
	Abweichung & $0\%$ & $1,5\%$ & $3,8\%$ \\ \hline
	\end{tabular}
\label{tab1}
\end{table}

\subsection{Koinzidenztheorie}
Eine Koinzidenz ist dann vorhanden, wenn zwei Detektoren
innerhalb eines Zeitintervalls, der Koinzidenzzeit, beide ein 
Signal messen. Unterschieden werden muss dabei zwischen den 
wahren Koinzidenzen und den zufälligen Koinzidenzen. Bei den wahren
Koinzidenzen können die beiden Signale einem einzigen physikalischen
Prozess zugeordnet werden. Die zufälligen Koinzidenzen entstehen
durch Detektion zweier Teilchen, die durch unabhängige physikalische
Vorgänge innerhalb de Koinzidenzzeit emittiert wurden. Im Experiment
soll erreicht werden, dass das Verhältnis von wahren zu zufälligen 
Koinzidenzen möglichst hoch ist, was durch geeignetes Wählen der 
Versuchsbedingungen erreicht wird.

Des Weiteren wird eine möglichst hohe Koinzidenzrate angestrebt. 
Die Einzelzählraten der beiden Detektoren ergeben sich aus
\begin{equation}
 Z_i=\varepsilon_i \omega_i Q, \;\;\; i=1,2,
\end{equation}
wobei $\varepsilon_i$ die Ansprechempfindlichkeit ist, $\omega_i$
der Raumwinkel, unter dem der Detektor die Signalqualle sieht und
$Q$ die Zerfallsrate der Probe. 

Um die Koinzidenzrate zu erhalten, muss die Zählrate des 1. Detektors
mit der Wahrscheinlichkeit multipliziert werden, dass der 2. 
Detektor ein Signal registriert. Diese Wahrscheinlichkeit kann 
vom Winkel abhängen, unter dem beide Signale gemessen werden, daher
wergibt sich für die Wahrscheinlichkeit einer Koinzidenz mit der 
Winkelverteilung $W(\vartheta)$:
\begin{equation}
 P(\vartheta)=\epsilon_2\omega_2 W(\vartheta)
\end{equation}
Die echte Koinzidenzrate ist daher:
\begin{equation}
 Z_{eK}=\varepsilon_1 \omega_1 \varepsilon_2 \omega_2 Q W(\vartheta)
\end{equation}

Um die Rate der zufälligen Koinzidenzen zu erhalten, betrachtet man die Wahrscheinlichkeit, dass
nach dem Messen von Detektor 1 innerhalb der Koinzidenzzeit $\tau$ ein unkorreliertes Signal im Detektor 2 
gemessen wird: $Z_2\cdot \tau$

Die Rate zufälliger Koinzidenzen ist daher:
\begin{equation}
 Z_{zK}=\tau Z_1Z_2
\end{equation}
Und das Verhältnis von echten zu zufälligen Koinzidenzen ist dann:
\begin{equation}
 \frac{Z_{eK}}{Z_{zK}}=\frac{1}{\tau Q}
\end{equation}
Für die Versuchbedingungen bedeutet das, die Koinzidenzzeit so gering wie möglich zu wählen.
Außerdem sollten eine radioaktive Probe genutzt werden, deren Zerfallsrate nicht zu hoch ist,
aber dennoch nicht so gering,
dass man mit einem akzeptablen Zeitaufwand ein statistisch relevantes Ergebnis erhält.

\subsection{$\gamma$-$\gamma$-Winkelkorrelation}
Entstehen bei einem radiaktiven Zerfall oder einer Annihilation zwei Gamma-Quanten, sind sie
in ihrem Winkel zueinander korreliert. Das heißt, dass es eine Winkelverteilung der Intensität
der Koinzidenzen gibt, die charakteristisch für diese Art Zerfall ist. 

Der einfachste Fall ist die Annihilation von Positron und Elektron, bei der Gesamtimpuls von $~0$ 
erhalten bleibt. So werden die Gamma-Quanten unter einem Winkel von $180^\circ$ zueinander gemessen.

Im Fall eines Kerns, der über eine Kaskade von 2 Gamma-Quanten zerfällt wird die Annahme gemacht, dass
die Lebensdauer des Zwischenzustands so kurz ist, dass sich die Orientierung des Kerns im Vergleich 
zum ersten Zerfall nicht verändert. 

Allgemein kann die Winkelverteilung über die Formel
\begin{equation}
 W(\theta)=\sum_\nu A_\nu^{(1)} A_\nu^{(2)} P_\nu ( \cos \theta)
\end{equation}
angegeben werden. Dabei ind die $A_\nu^{(i)}$ Koeffizienten, die aus Tabellen abgelesen werden können.
$P_\nu(\cos \theta)$ sind Legendre-Polynome der $\nu$-ten Ordnung, wobei für nicht-polarisierte Strahlung
allerdings nur geradzahlige Polynome in der Winkelverteilung auftreten. Dadurch treten nur geradzahlige
Potenzen von $\cos \theta$ auf, was die Winkelverteilung $W(\theta)$ zu einer zu $90^\circ$ symmetrischen
Funktion macht. Es reicht daher die Messung auf einen Bereich von $\theta=0^\circ$ bis $90^\circ$ zu 
beschränken.

\section{Versuchsaufbau und Messgeräte}

\begin{align}
U_{eff} = \frac{U_0}{\sqrt{2}} = \frac{\frac{U_{PeakPeak}}{2}}{\sqrt{2}} = \frac{17,4 V}{ 2 \sqrt{2}} \approx 6,15 V
\end{align}


%\begin{figure}[htbp] 
%     \includegraphics[width=0.5\textwidth]{Lissajous.jpg}
%  \caption{Beispielfigur}
%  \label{Lissajous}
%\end{figure}


\section{Versuchsteil 1}


\subsection{Durchführung}


\subsection{Auswertung}


\section{Versuchsteil 2}



\subsection{Durchführung}



\subsection{Auswertung}




\section{Zusammenfassung/Fazit}

Text


\end{document}