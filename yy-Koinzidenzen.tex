\documentclass[11pt]{scrartcl}

\usepackage{ucs}
\usepackage[utf8x]{inputenc}
\usepackage[T1]{fontenc}
\usepackage[ngerman]{babel}
\usepackage{amsmath,amssymb,amstext}
\usepackage{graphicx}
\usepackage[justification=RaggedRight, singlelinecheck=false]{caption}


\title{Fortgeschrittenen Praktikum Teil 1: IKF}
\subtitle{Versuch 1: $\gamma \gamma$ - Koinzidenzen \\ Betreuer: Alexander Schottelius}

\author{Gruppe 1: Reinhold Kaiser, Florian Stoll}

\date{14.04.2018}

\begin{document}
\maketitle
\newpage
\tableofcontents
\newpage


\section{Zielsetzung}


\section{Theoretische Grundlagen}

\subsection{Betazerfall von $^{22}Na$ und $^{60}Co$}

\subsection{Wechselwirkungen elektromagnetischer Strahlung mit Materie}

\paragraph{Der Photo-Effekt}



\paragraph{Der Compton-Effekt}



\subsection{Die Paarvernichtung}
Wie jedes Elementarteilchen besitzt auch das Elektron ein Anti-Teilchen mit gleichem Spin und gleicher Ruhemasse, jedoch mit entgegengesetzter Ladung. Dieses wird als Positron bezeichnet. Als Formelzeichen verwendet man hier $ e^{+} $ um die positive Ladung kennzuzeichnen. Dieses Positron kann als Nebenprodukt bei radioaktiven Atomkernen ( $\beta {^+} $ - Zerfall) oder durch die Paarbildung eines $\gamma $- Quants, das in ein Positron und eine Elektron zerfällt, entstehen. 
Die sogenannte Paarvernichtung ist der Umkehrprozess zur Paarbildung. Dabei wird ein Elektron-Positron-Paar wieder in elektromagnetische Strahlung umgewandelt. Dabei können jedoch unterschiedliche Anzahl an Photonen entstehen. Auch hier muss aber ein Stoß vorliegen, da sonst Impuls- und Energieerhaltung verletzt werden würden. Ein sogenannter Ein-Quanten-Zerfall, bei dem ein Paar tatsächlich nur in ein Photon umgewandelt wird, kann aus den gleichen Gründen nur bei einem Stoß mit einem Impuls- und Energieaufnahmefähigen Partner vorkommen, was nur im Festkörper vorkommt. Dieser Zerfall ist allerdings sehr unwahrscheinlich, und wird für diesen Versuch auch nicht berücksichtigt. 
Da das Elektron und das Positron beide den Spin $\frac{1}{2}$ haben, gibt es zwei verschiedene Einstellmöglichkeiten der Spins zueinander. Liegen die Spins parallel, besitzt das Paar einen Gesamtspin von 1, liegen sie anti-parallel ergibt sich ein Gesamtspin von 0. Der Grundzustand mit $S=0$ ist dabei nicht entartet, der Zustand mit $S=1$ ist dreifach entartet. Daraus ergibt sich eine Wahrscheinlichkeit von 3:1 für einen Zerfall des Triplett-Zustands, da die Wahrscheinlichkeiten für die Einstellmgölichkeiten der beiden Spins alle gleich sind. Die bei der Paarvernichtung entstehenden Quanten haben allerdings den Spin 1. Darum kann der Grundzustand nur in zwei $\gamma$ - Quanten zerfallen, während der Triplett-Zustand in drei Photonen zerfallen muss. Das stellt allerdings einen Prozess höherer Ordnung dar und ist daher deutlich unwahrscheinlicher. Im durchgeführten Versuch werden wir uns auch ausschließlich auf Paarvernichtungen aus dem Singulett - Zustand, d.h. mit zwei auftretenden $\gamma$ - Quanten beschränken.


Text 

\begin{table}
	\caption{Beispieltabelle}
	\begin{tabular}{|l|l|l|l|}
	\hline
	 & Netzteil & Multimeter & Oszilloskop \\ \hline
	Spannung & $5V$ & $ 5,076 V $ & $ 5,19 V $ \\ \hline
	Abweichung & $0\%$ & $1,5\%$ & $3,8\%$ \\ \hline
	\end{tabular}
\label{tab1}
\end{table}

\section{Versuchsaufbau und Messgeräte}

\begin{align}
U_{eff} = \frac{U_0}{\sqrt{2}} = \frac{\frac{U_{PeakPeak}}{2}}{\sqrt{2}} = \frac{17,4 V}{ 2 \sqrt{2}} \approx 6,15 V
\end{align}


%\begin{figure}[htbp] 
%     \includegraphics[width=0.5\textwidth]{Lissajous.jpg}
%  \caption{Beispielfigur}
%  \label{Lissajous}
%\end{figure}


\section{Versuchsteil 1}


\subsection{Durchführung}


\subsection{Auswertung}


\section{Versuchsteil 2}



\subsection{Durchführung}



\subsection{Auswertung}




\section{Zusammenfassung/Fazit}

Text


\end{document}