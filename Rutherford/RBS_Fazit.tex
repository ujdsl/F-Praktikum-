\section{Zusammenfassung/Fazit}
Zusammenfassend lässt sich sagen, dass die Funktionsweise und die Messmethoden eines van-de-Graaf-Beschleuniger im durchgeführten Versuch sehr gut durchführ- und nachvollziehbar war. Es konnten einige alltagsnahe und gute Versuchsergebnisse erzielt werden, so hat die Element-Bestimmung z.B. sehr gut geklappt. Die Versuchsdurchführung war trotz des komplizierten Aufbaus einfach zu gestalten und es konnten einige interessante Erkenntnisse über die Arbeit an großen Beschleunigern gewonnen werden. Um noch genauere Auflösungen z.B. des Materials der Kupfermünze zu erreichen, müsste natürlich die Strahlenergie erhöht werden, wobei das hier erreichte Auflösungsniveau für den Praktikumsversuch absolut ausreichend ist. 