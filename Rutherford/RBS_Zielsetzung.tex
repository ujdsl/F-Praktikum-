\section{Zielsetzung}
In diesem Versuch geht es darum, die Rutherford-Backscattering-Spectrometry (RBS) an verschiedenen Proben durchzuführen. In der RBS wird meist mit Wasserstoff- oder Helium-Kernen mit einer Energie von $1-4MeV$ auf ein Material geschossen, wobei die Eindringtiefe bis circa $1\mu m$ reicht \cite{anleitung}. Leichte Kerne mit geringerer Energie hätten nicht so hohe Eindringtiefen und nach oben hin ist kinetische Energie der Projektile dadurch begrenzt, dass irgendwann Kernreaktionen eine Rolle spielen.

Aufgrund dieser Eigenschaften eignet sich die RBS dazu, Oberflächen und dünne Filme auf unterschiedliche Eigenschaften zu testen. Beispielsweise kann die Schichtdicke einer oder mehrer Schichten bestimmt werden. Außerdem ist es möglich, Fehlstellen in Kristallen zu lokalisieren. In diesem Versuch soll allerdings nur das elementarste Verfahren, nämlich die Elementidentifikation angewendet werden. 

Zur Beschleunigung der He$^+$-Ionen wird ein Van-de-Graaff-Beschleuniger verwendet, wobei die kinetische Energie der Ionen bekannt ist. Mit bekannten Targets wird zunächst die Kalibrierung des Detektors durchgeführt. Und im eigentlichen Teil des Versuchs sollen dann die Elemente mehrerer unbekannter Proben bestimmt werden.
