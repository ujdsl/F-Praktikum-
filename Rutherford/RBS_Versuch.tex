\section{Durchführung und Auswertung}
Nachdem im letzten Versuchsteil die Kalibrierung der Energie über die Kanäle bestimmt wurde, werden nun sechs verschiedene, unbekannte Proben mithilfe der Rutherford-Backscattering-Methode auf ihre chemische Zusammensetzung analysiert. Aus den gemessenen Werten für die Rückstreuungsenergie kann so der K-Wert berechnet und mit den bekannten theoretischen K-Werten verglichen werden. Daraus lassen sich so die beinhalteten Elemente bestimmen. Die gemessenen Kanäle, Energien und K-Werte zusammen mit der Element-Bestimmung finden sich in den Tabellen \ref{Probe1} bis \ref{Probe6}. 

\begin{table}[h]
	\caption{Auswertung RBS - Probe 1}
	\begin{tabular}{|c|c|c|c|c|c|}
	\hline
	 & Mittl.Kanallage & Energie in keV & K-Faktor exp. & K-Faktor theo. & Element \\ \hline
	   1 & 692 & 1431 & 0,712 & 0,716 & Ti\\ \hline
	   2 & 529 & 1101 & 0,551 & 0,552 & Al \\ \hline
	\end{tabular}
\label{Probe1}
\end{table}

\begin{table}[h]
	\caption{Auswertung RBS - Probe 2}
	\begin{tabular}{|c|c|c|c|c|c|}
	\hline
	 & Mittl.Kanallage & Energie in keV & K-Faktor exp. & K-Faktor theo. & Element \\ \hline
	   1 & 818 & 1687 & 0,843 & 0,842 & Nb\\ \hline
	   2 & 693 & 1434 & 0,717 & 0,716 & Ti \\ \hline
	   3 & 526 & 1095 & 0,548 & 0,552 & Al \\ \hline
	\end{tabular}
\label{Probe2}
\end{table}

\begin{table}[h]
	\caption{Auswertung RBS - Probe 3}
	\begin{tabular}{|c|c|c|c|c|c|}
	\hline
	 & Mittl.Kanallage & Energie in keV & K-Faktor exp. & K-Faktor theo. & Element \\ \hline
	   1 & 532 & 1107 & 0,550 & 0,552 & Al\\ \hline
	   2 & 343 & 725 & 0,362 & 0,362 & O \\ \hline
	\end{tabular}
\label{Probe3}
\end{table}

\begin{table}[h]
	\caption{Auswertung RBS - Probe 4}
	\begin{tabular}{|c|c|c|c|c|c|}
	\hline
	 & Mittl.Kanallage & Energie in keV & K-Faktor exp. & K-Faktor theo. & Element \\ \hline
	   1 & 650 & 1346 & 0,673 & 0,674 & Ca\\ \hline
	   2 & 403 & 846 & 0,423 & 0,427 & F \\ \hline
	   3 & 237 & 510 & 0,255 & 0,252 & C \\ \hline
	\end{tabular}
\label{Probe4}
\end{table}

\begin{table}[h]
	\caption{Auswertung RBS - Probe 5 (Münze)}
	\begin{tabular}{|c|c|c|c|c|c|}
	\hline
	 & Mittl.Kanallage & Energie in keV & K-Faktor exp. & K-Faktor theo. & Element \\ \hline
	   1 & 745 & 1539 & 0,769 & 0,778 & Cu\\ \hline
	\end{tabular}
\label{Probe5}
\end{table}

\begin{table}[h]
	\caption{Auswertung RBS - Probe 1}
	\begin{tabular}{|c|c|c|c|c|c|}
	\hline
	 & Mittl.Kanallage & Energie in keV & K-Faktor exp. & K-Faktor theo. & Element \\ \hline
	   1 & 624 & 1294 & 0,647 & 0,637 & Cl\\ \hline
	   2 & 484 & 1010 & 0,505 & 0,497 & Na \\ \hline
	\end{tabular}
\label{Probe6}
\end{table}

Besonders auffällig stellt sich unsere Probe 5 dar, welche aus einer 1-Cent-Euro-Münze bestand. Laut \cite{muenze} bestehen die Centmünzen aus Stahl mit einer Kupferauflage. Die Kupferauflage können wir mit unserem Experiment sehr gut bestätigen, während der Hinweis auf den Stahlbestandteil der Münze ausbleibt. Wir vermuten, dass die vorhandene Kupferschicht wohl zu dick ist, sodass keine der Strahlen den Stahlkern erreichen. hier könnten wir die Energie des Strahls erhöhen, um eine höhere Eindringtiefe zu erreichen. 

