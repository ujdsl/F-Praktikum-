\section{Zielsetzung}
In diesem Versuch geht es darum, verschiedene Proben mit der Methode der magnetischen Kernresonanz, auf englisch Nuclear Magnetic Resonance (NMR), zu untersuchen. Die NMR ist die herausragende Technik um die Struktur organischer Komponenten zu bestimmen \cite{anleitung}. Aber auch andere Proben, die molekularer oder kristalliner Form sind, können mit Hilfe der NMR untersucht werden. Der große Vorteil der NMR ist, dass die Probe bei der Untersuchung nicht zerstört wird, weswegen sie auch Anwendung in der Medizin findet.

In unserem Fall wird die NMR dazu genutzt, für zwei unterschiedliche Proben (Kupfersulfat und Teflon) die gyromagnetischen Verhältnisse und die Landé-g-Faktoren zu bestimmen. So kann auch geklärt werden, welcher Kern oder welche Kerne der Probe auf die NMR reagieren. Das Magnetfeld, was an die Proben angelegt wird, muss dafür kalibriert werden. Es wird Glycerin untersucht, dass ein bekanntes gyromagnetisches Verhältnis hat, woraus dann auf das Magnetfeld an der Probe geschlossen werden kann.
