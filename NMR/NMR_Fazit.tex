\section{Zusammenfassung/Fazit}
In diesem Versuch wurden mit Hilfe der NMR-Spektroskopie mehrere Proben untersucht und die gyromagnetischen Verhältnisse sowie die Landé-g-Faktoren bestimmt. Im Vergleich mit den Literaturwerten konnten einerseits die zugehörigen Kernmomente identifiziert werden und andererseits stellte sich heraus, dass diese Messmethode sehr genau ist, da die Literaturwerte bis auf einen kleinen Fehler gut reproduziert wurden. Ein erforderliches Mittel für diese genauen Werte waren die 20-fache Wiederholung der Messung für jede Probe.

Insgesamt wurde ein guter Überblick gewonnen, wie die NMR-Spektroskopie funktioniert und welche Herausforderungen dabei auftreten können. Es wurde auch gezeigt, welch großer Anwendungsbereich die NMR-Spektroskopie hat und wie vielfältig sie eingesetzt werden kann.
