\section{Zusammenfassung/Fazit}
Im durchgeführten Versuch konnte der Hall-Effekt sehr anschaulich gezeigt und auf die Bandstruktur von Halbleitern angewandt werden. Dabei waren sowohl die angewandten experimentellen Methoden als auch die gezeigten theoretischen Zusammenhänge mit zufriedenstellendem Erkenntnisgewinn verbunden. Die Van-der-Pauw-Methode stellte dabei eine interessante Alternative zur Vermessung des Hall-Effekts dar. Besonders die vermessenen Temperaturverläufe der Ladungsträgerdichte und der Beweglichkeit sorgten für eine willkommene Veranschaulichung der bekannten theoretischen Verläufe. Die Bestimmung der Bandlücke war nur sehr fehlerbehaftet möglich, die Idee dahinter konnte jedoch vermittelt und beschrieben werden.