\section{Theoretische Grundlagen}
\subsection{Bandstrukturen}
Bandstrukturen in Kristallen sind dadurch zu erklären, dass sich die Atomorbitale der einzelnen Atome überlagern. Aus den diskreten Energieniveaus der einzelnen Atome werden so Energiebänder. Dieses Modell nennt man auch ``tight binding''. Ein anderes Modell genannt ``plane wave expansion'' nimmt an, dass die Elektronen im Kristall quasi frei sind und so durch ausgedehnte Wellen beschrieben werden können. Die Dispersionsrelation wird aber durch die Gitterstruktur verändert und so entstehen Energiebänder, die so Bandlücken in die eigentlich verbundene Energieverteilung bringen.

Anhand der Besetzung dieser Energiebänder lassen sich Materialien in drei unterschiedliche Kategorien bezüglich der Leitfähigkeit einordnen. Denn ein Material leitet nur, wenn das energetisch höchste Energieband nur teilweise besetzt ist. Im Gegensatz zu diesen Metallen haben die Isolatoren zwischen dem letzten besetzten Energieband (Valenzband) und dem nächsten unbesetzten Energieband (Leitungsband) eine Bandlücke, die von den Elektronen normalerweise nicht überwunden werden kann. Bei den Halbleitern ist diese Bandlücke besonders klein, sodass Elektronen beispielsweise durch thermische Anregung in das Leitungsband gelangen können. Die Größe der Energielücken für Halbleiter ist im Bereich von einigen Zehntel eV.

Intrinsische Halbleiter sind Halbleiter, die keine Fehlstellen oder Fremdatome haben und dennoch durch elektronische Anregungen leitfähig werden können. Um nun die Konzentration der Elektronen im Leitungsband zu berechnen nutzt man die Beziehung
\begin{equation}
\frac{N}{V}=n = 2\cdot \int_{E_L}^\infty D_L(E)\cdot f(E)dE
\end{equation}
mit der Fermi-Verteilung $f(E)$ und der Zustandsdichte im Leitungsband $D_L(E)$.
Analog findet man die Beziehung für die Konzentration der Löcher im Valenzband:
\begin{equation}
 \frac{P}{V}=p = 2\cdot \int_{E_V}^\infty D_V(E)\cdot [1-f(E)]dE
\end{equation}
Dazu muss gesagt werden, dass Löcher im Valenzband einfach fehlende Elektronen darstellen. Sie entstehen, indem Elektronen angeregt werden und aus dem Valenzband ins Leitungsband gehen. Das Konzept des Lochs ist wichtig, weil sie auch zum Ladungstransport beitragen, indem die anderen Valenzelektronen in die Löcher nachrücken. Effektiv bewegt sich dann das Loch und transportiert so positive Ladung.

Für die Dispersionsrelation für Elektronen im Leitungsband bzw. Löcher im Valenzband wird
\begin{equation}
 E(\vec k)= E_L + \frac{\hbar}{2}\cdot \left(\frac{k_x^2}{m^*_{xx}}\cdot \frac{k_y^2}{m^*_{yy}}\cdot\frac{k_z^2}{m^*_{zz}} \right)
\end{equation}
angenommen und dies führt im Fall isotroper effektiver Massen zu den Zustandsdichten
\begin{eqnarray}
 D_L(E)&=&\frac{1}{4\pi^2}\left(\frac{2m_n^*}{\hbar^2}\right)^\frac{3}{2}(E-E_L)^\frac{1}{2},\;\; E>E_L\\
 D_V(E)&=&\frac{1}{4\pi^2}\left(\frac{2m_p^*}{\hbar^2}\right)^\frac{3}{2}(E_V-E)^\frac{1}{2},\;\; E_V>E.
\end{eqnarray}


\subsection{Dotierung von Halbleitern}
- Dotierung (9-10)
- Temperaturabhängigkeit der Ladungsträgerdichte (11-13)

\subsection{Ladungstransport in Halbleitern}
- Drude Modell (23)
- Streuprozesse im Halbleiter und Beweglichkeit (14-16)

\subsection{Hall-Effekt}
Der Hall Effekt tritt auf, wenn in einem Leiter oder einem Halbleiter senkrecht zu einem Magnetfeld $\vec B$ ein Strom fließt. Die Lorentzkraft
\begin{equation}
 m_e\frac{\mathrm d v}{\mathrm d t}= -e(\vec E + \vec v \times \vec B)
\end{equation}
lenkt dann die Ladungsträger so ab, dass ein elektrisches Feld $E_H$ entsteht, das zu Magnetfeld $B$ und Strom $j_x$ senkrecht steht. Im Folgenden fließe der angelegte Strom in x-Richtung, das Hall-Feld zeige in y-Richtung und das Magnetfeld in z-Richtung. Nun definiert man die Hall-Konstante durch 
\begin{equation}
 R_H=\frac{E_H}{j_x\cdot B}.
\end{equation}

$j$ wird durch die Konzentrationen $n,p$ und die Driftgeschwindigkeiten $v_e, v_h$ der Löcher und Elektronen ausgedrückt:
\begin{equation}
 \vec j=-ne\vec v_e+pe\vec v_h
\end{equation}
und wiederum die Driftgeschwindigkeiten durch 
\begin{eqnarray}
 v_x^e &=& -\mu_eE_x+\mu_e\omega_e\tau_eE_H\\
 v_x^h &=& +\mu_hE_x+\mu_h\omega_h\tau_hE_H\\
 v_y^e &=& -\mu_eE_H-\mu_e\omega_e\tau_eE_x\\
 v_y^h &=& +\mu_hE_H-\mu_h\omega_h\tau_hE_x.
\end{eqnarray}
Dabei sind die $\mu$ die Beweglichkeiten der Löcher bzw. Elektronen, die $\omega=\frac{eB}{m}$ die Zyklotronfrequenzen und die $\tau$ die Relaxationszeiten der Elektronen bzw. Löcher. Dabei  wurden Terme der Ordnung $\omega^2$ vernachlässigt, was äquivalent zur Vernachlässigung des Magnetoresistiven Effekts ist. Der Magnetoresistive Effekt bezeichnet das Ändern des elektrischen Widerstands eines Materials durch Anlegen eines Magnetfelds.

Setzt man nun den statischen Fall als Voraussetzung, ist also $j_y=0$, dann ergibt sich durch Einsetzen der Driftgeschwindigkeiten und elektrischer Felder die Hall-Konstante zu
\begin{equation}
 R_H = \frac{1}{e}\cdot\frac{p-n\left(\frac{\mu_e}{\mu_h}\right)^2}{\left(p+n\frac{\mu_e}{\mu_h}\right)}
\end{equation}
