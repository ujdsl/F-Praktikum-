\section{Theoreticals}
\subsection{Energy States of Rubidium and Zeemann-Effect}

According to rules of Quantum Mechanics, every atom has its own quantized energy states, in which the electrons are positioned. The states are described by some quantum numbers, which are called n, l and s. n is the energy quantum number, l describes the angular momentum and s shows the electron spin and obviously is $s=\frac{1}{2}$. We notate these states as 1S, 2S, 2P and so on.
The angular momentum and spin couple to the so-called finestructur. This are some energy corrections to the states, which also can split up through this coupling. Mathematically, we add the quantum numbers l and s to the total angular momentum j. Because the directions of l and s can be the same or different, we get in the case of l=1 two different possibilities for J, which are generated by $J=\frac{1}{2}$ and $J=\frac{3}{2}$ and have different energy niveaus. Our notations changes so to $1S_{\frac{1}{2}}, 2S_{\frac{1}{2}}, 2P_{\frac{1}{2}}, ...$.

For the complete correct energy niveaus, we have to add another quantum number, which describes the spin of the atomcore. It is called I and counts $I=\frac{3}{2}$ in our case, as we work with $^{87}Rubidium$. The angular momentum I therefore adds to the total angular momentum J to another total angular momentum called F. The resulting energy states are called hyperfinestructur.

In Figure \ref{states} are plotted the S and P states of Rb. Addiotionally there are marked the wavelenghts  for the energy difference of the different states and the magnetic quantum number m, which always appear with an angular momentum quantum number. In our case, the resulting angular momentum quantum number is F and m always runs from -F to F, so the figure can be understood well. Without external fields, the energy of all states with different values of m is exactly the same, so usually there is no such splitting seen.

\begin{figure}[htbp] 
     \includegraphics[scale=0.7]{level87.png}
  \caption{Energy states of $^{87}Rb$}
  \label{states}
\end{figure}

The existence of these magnetic quantum numbers is responsible for the so-called zeeman-effect, which appears if we put our atom in a small magnetic field. It is possible to calculate the energy correction effect of the magnetic coupling of external field and orbital angular momentum. The correction counts $E_Z=g_F\mu_0 Bm$
According to that, we now have different energx levels for different values of m. Our old energy states is so split up in $2F+1$ states with different energies, which are separated by the same energy difference. $g_F$ is the Landé-factor, which is different in every state of the hyperfinestrucutr can be calculated by:

\begin{align}
g_F=g_J \frac{F\left(F+1\right)+J\left(J+1\right)-I\left(I+1\right)}{2F\left(F+1\right)}
\end{align}

And continued:

\begin{align}
g_J=\frac{J\left(J+1\right)+L\left(L+1\right)-S\left(S+1\right)}{2J\left(J+1\right)}
\end{align}