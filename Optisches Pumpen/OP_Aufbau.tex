\section{construction of experiment and used devices}
Central part of our experiment is a cell with the rubidium gas in it, which we want to investigate. We also need a lamp, from which we radiate the light to the rubidium cell. In the direct opposite of the lamp, we have our detector to measure the incoming and therefore not absorbed radiation. Of course, we have to use polarized light for optical pumping, while the light of the lamp is unpolarized. So we have two filters in the beam between the lamp and the Rb-cell. The first filter polarizes the incoming light linearly, so it can be polarized circular more easy. That can be reached by the second filter, which is a so-called $\frac{\lamda}{4}$-plate. This is mainly a cristal with a special grid distance, which delays one part of the radiation wave and has no effect to the other part. After that filter, we have a right-handed circular polarized light, as we need to realize optical pumping. Also, our beam is focused twice by lenses, once for aiming the radiation in the Rb-cell and afterwards to be focused in the detector. 

In our experiment, we need different magnetic fields, so we have two pairs of helmholtz coils aroung the rubidium cell. One of it generates a field in vertical direction, while the other pair generates a horizontal field. The coil pairs are driven  either by a programmable DC power supply or a function generator for AC currents. Depending on the connected device, we can generate constant or oscillating magnetic fields. 

Additionally, we heat our Rb-cell with a little heating system, so the Rb-atoms are completely in gas state. 

